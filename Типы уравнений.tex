\documentclass[a4paper, 12pt]{article}
\usepackage[english, russian]{babel}
\usepackage{listings}
\usepackage[top=2cm, bottom=2cm, left=2cm, right=2cm, bindingoffset=0pt]{geometry}
\usepackage{hyperref}
\usepackage{amsmath,amsfonts,amssymb,amsthm,mathtools}
\usepackage[T2A]{fontenc}
\usepackage[utf8]{inputenc}

\title{Типы уравнений}
\author{Бирюк Илья Александрович}

\begin{document}
\maketitle
\newpage
\tableofcontents
\newpage
\section{Проверка}
Дано: $x=f(t)$ 
Пример: $x=t/2$

$D^2x+2Dx+1=3t$, $D^2x = x''$
\begin{enumerate}
    \item Задаём $I$, по $x$. Так как в примере $t$ сверху, то $I=(-\infty, +\infty)$
    \item Подставляем $x$ в уравнение.
    $0+1+1=3$
\end{enumerate}
\section{Просто решение уравнения c нулём на конце}
Дано: $D^4x-7D^3x+17D^2x-17Dx+6=0$
\begin{enumerate}
    \item Перепишем $D^nx=\lambda^n$
    
    $\lambda^4-7\lambda^3+17\lambda^2-17\lambda+6=0$
    \item Находим корни(решаем)
    \begin{enumerate}
        \item $\lambda_1=1,k=2$, $k$ --- Степень корня
        \item $\lambda_2=2,k=1$
        \item $\lambda_1=3,k=1$
    \end{enumerate}
    \item Перепишем уравнение в $L_n$ виде
    
    $(D-1D^0)^2(D-2D^0)(D-3D^0)x=0$
    \item Записываем $x_i$
    
    $x_1=(C_0+C_1t)e^{1t}$

    $x_2=C_2e^{2t}$

    $x_3=C_3e^{3t}$
    \item Сумируем $x_i$
\end{enumerate}
\section{Проверить является ли эти $\phi$ базисом}
\begin{enumerate}
    \item Строим вронскиан ($w(t)$)
    
    $
    \begin{bmatrix}
        \varphi_1 & \varphi_2 & \cdots & \varphi_N \\
        D\varphi_1 & D\varphi_2 & \cdots & D\varphi_N \\
        D^2\varphi_1 & D^2\varphi_2 & \cdots & D^2\varphi_N \\
        D^3\varphi_1 & D^3\varphi_2 & \cdots & D^3\varphi_N \\
        \vdots & \vdots & \ddots & \vdots \\
    \end{bmatrix}
    $

    \item Проверяем его на базис, беря любое удобное $t$ (обычно равное $0$)
\end{enumerate}
\section{Метод вариации произвольных постоянных (Правило Лагранжа)}
Дано: $D^2x-2Dx+x=e^t/(t^2+1)$
\begin{enumerate}
    \item Решаем $\lambda$

    $\lambda^2-2\lambda+1=0$
    
    $\lambda_1=1,k=2$
    \item Выводим $X_{oo}$
    
    $X_{oo}=(C_0+C_{1}t)e^{t}$

    Идеал: $X_{oo}=\sum_{i=1}^{n}(C_0+C_1t+\dots+C_{k_{i}-1}t^{k_i-1})e^{\lambda_i t}$

    \item Составляем систему уравнений
    
    Идеал: 
    \[
\left\{
\begin{aligned}
    \psi_0 D u_0 + \dots + \psi_{n-1} D u_{n-1} &= 0, \\
    D \psi_0 D u_0 + \dots + D \psi_{n-1} D u_{n-1} &= 0, \\
    & \dots \\
    D^{n-2} \psi_0 D u_0 + \dots + D^{n-2} \psi_{n-1} D u_{n-1} &= 0, \\
    D^{n-1} \psi_0 D u_0 + \dots + D^{n-1} \psi_{n-1} D u_{n-1} &= f.
\end{aligned}
\right.
\]

Для того чтобы получить какие-либо $\psi$ Надо взять $X_{oo}$ и подставить некоторые, базисные $C$
\[
\left\{
\begin{aligned}
    \psi_0 = [C_0 = 1, C_1 = 0] = (1+0*t)e^{t} =  e^{t} \\
    \psi_1 = [C_0 = 0, C_1 = 1] = (0+1*t)e^{t} = te^{t}
\end{aligned}
\right.
\]
    \item Решаем систему
    
\[
\left\{
\begin{aligned}
    e^{t} D u_0 + te^{t} D u_{1} &= 0, \\
    e^{t} D u_0 + (e^{t}+te^{t}) D u_{1} &= \frac{e^{t}}{t^2+1}, \\
\end{aligned}
\right.
\]    

    \item Подставляем
    $$x_{\text{чн}}(t) = \sum_{k=0}^{n-1} u_k(t) \psi_k(t)$$
    \item Складываем
    $x=x_{\text{оо}}+x_{\text{чн}}$
\end{enumerate}

Если даны условия коши, то их подставляем в конце.
\section{Правило Коши}
Дано:

$D^2x+2Dx+x = e^{2t}$

$x|_{t=1}=1, Dx|_{t=1}=5$
\begin{enumerate}
    \item Решаем $\lambda$
    
    $\lambda^2+2\lambda+1=0$

    $(\lambda+1)^2=0$

    $\lambda=-1,k=2$
    \item Выводим $X_{oo}$
    
    $X_{oo}=(C_0+C_{1}t)e^{-t}$

    $D_tX_{oo}=C_1e^{-t}-(C_0+C_{1}t)e^{-t}$
    \item Решаем систему
    
\[
\left\{
\begin{aligned}
    X_{oo}(1)=1 \\
    DX_{oo}(1)=5 \\
\end{aligned}
\right.
\]  
\[
\left\{
\begin{aligned}
    \frac{(C_0+C_{1})}{e}=1 \\
    \frac{C_0}{e}=5 \\
\end{aligned}
\right.
\]  
\[
\left\{
\begin{aligned}
    C_1=-6e \\
    C_0=5e \\
\end{aligned}
\right.
\] 
    \item Находим $\varphi$
    
    Идеал: 

    $\varphi(t)=(-5e+6et)e^{-t}$

    \item Решаем $x_{\text{чн}}=\int_{0}^{t}(-5e+6e(t-\tau))e^{-(t-\tau)}d\tau$
    \item $x=x_{oo}+x_{\text{чн}}$
\end{enumerate}
\section{Правило Эйлера}
$te^4$ - квазиполином

$D^2x-2Dx+x=6te^t$

\begin{enumerate}
    \item Находим контрольные числа(перед т). И м для них(порядок(степень перед е) многочлена) если косинус, то левее.
    
    $\gamma_1=1, m_1=1$

    При синусах и косинусах

    $(\sin t+t\cos t)e^{0t}\rightarrow\gamma=(\gamma\pm\beta i)$

    $1e^{-t}\cos2t\rightarrow\gamma=(\gamma\pm\beta i)=-1\pm2i, m=0$ (sin тоже)
    \item находим $\lambda$ и $X_{oo}(t)$
    
    $\lambda^2-2\lambda+1=0$

    $\lambda=1,k=2$

    $X_{\text{oo}_i}(t)=(C_0+C_1t)e^{t}$
    \item Для каждого контрольного числа смотрим совпадает ли оно с лямбдой, если да r = k, иначе r = 0(ни с одним)
    
    да
    \item $X_{\text{чн}_i}(t)=t^r(\text{многочлен m степени})e^{\gamma it}$
    
    $X_{\text{чн}_i}(t)=t^2(C_0+C_1t)e^{t}$

    Для комплексных

    $\gamma=-1\pm2i\rightarrow X_{cn}=t^r(C_0\sin2t+C_1\cos2t)e^{-t}$

    \item $X_{\text{чн}}(t)=\sum X_{cn_i}(t)$ - \textbf{общий вид} (содержит константы)
    
    $X_{\text{чн}_i}(t)=t^2(C_0+C_1t)e^{t}$
    \item Находим производные уравнения ($DX_{cn}, D^2X{xn}\dots$) и решаем как многочлен.
    
    $DX_{\text{чн}}=2C_0te^t+C_0t^2e^t+3C_1t^2e^t+C_1t^3e^t$

    $D^2X_{\text{чн}}=2C_0e^t+4C_0te^t+C_0t^2e^t+6C_1te^t+3C_1t^2e^t+3C_1t^2e^t+C_1t^3e^t$

    $2C_0e^t+4C_0te^t+C_0t^2e^t+6C_1te^t+3C_1t^2e^t+3C_1t^2e^t+C_1t^3e^t-(2C_0te^t+C_0t^2e^t+3C_1t^2e^t+C_1t^3e^t)+C_0t^2e^t+C_1t^3e^t=6te^t$

    $C_1-2C_1+C_1=0$

    $C_0+3C_1+3C_1-2C_0-6C_1+C_0=0$

    $2C_0+2C_0+6C_1+3C_1-4C_0=6$

    $C_1=1, C_0=0$
    \item подставляем $C_i$ в $X_{\text{чн}}$
    
    $X_{\text{чн}}=t^3e^t$
    \item сумируем $x=x_{oo}+x_{\text{чн}}$
\end{enumerate}
\end{document}